% Yiyong Sun,yiyognhit@gmail.com,2017.03.05 %% The current version should be listed at the first sight. Any changes, mark your name mailbox and the date

\documentclass[a4paper]{article}%{report}%{book}
%\newcommand{\CurrentDate}
\usepackage[section]{placeins}
\usepackage{longtable}
\usepackage{float}
\usepackage{url}
\usepackage{CJKutf8}
\usepackage{amsfonts}
\usepackage{amsmath}
\usepackage{mathrsfs}
\usepackage{latexsym}
\usepackage{graphicx}
\usepackage{indentfirst}%首行缩进
%\usepackage{exCJK}
%\usepackage[superscript]{cite}
%\usepackage[dvipdfmx,unicode,           %dvi-->pdf ????
            %bookmarksnumbered=true]{hyperref}
%\newtheorem{remark}{注解}

 % {gkai}{gbsn}

\begin{document}
\begin{CJK}{UTF8}{gbsn}
\title{高速高精度光机电一体化技术与设备}
\author{高会军
\thanks{\textbf{哈尔滨工业大学智能控制与系统研究所}}
\date{\CurrentDate} % Curent Date
}
\maketitle
\clearpage
\tableofcontents
\clearpage

\section{表面贴装技术背景及现状}
贴片机是一种高精尖光机电一体化设备,主要用来全自动高精度、高速地贴放电子元器件,是整个表面贴装(surface mount technoloty,简称SMT)生产线中最复杂、最核心的装备。随着中国电子制造业的高速发展,中国的SMT技术及产业也同时迅猛发展,表面贴装生产线的关键设备——贴片机在中国的保有量已位居世界前列,SMT贴片机市场占全球40\%。目前,中国SMT行业使用的主流中高速泛用贴片机仍然处于国外垄断状态,每年进口贴片机的开销达10亿美元以上,而中国自主研发的低俗专用贴片机由于产能地下、功能单一、稳定性差等原因,无法胜任大规模高速高精度的贴装任务。因此,国产高速高精度泛用贴片机的研发势在必行。
\subsection{起源}
\subsection{发展过程}
\subsection{表面贴装技术核心问题}
\subsection{前景}
\section{泛用贴片机研究意义}
市场需求大,但是依赖进口。国产低端。核心技术亟待突破。
\section{泛用贴片机的重难点问题}
\subsection{机械结构设计}
\subsection{机电系统设计}
\subsection{图形图像算法}
\subsection{优化算法设计}
\section{泛用贴片机国内外现状}
\subsection{国外研究进展}
\subsection{国内研制情况}
\section{国产贴片机的瓶颈}
\section{我们的优势}
\section{展望}
国家意义重大;

系列机电产品,填补国家空白;

与国际垄断厂商抗衡;

技术有借鉴意义,可以移植;

吸引高技术人才;

上下游产业链形成;

\begin{table}[h]
\caption{地址 HT-03-04-02-10 对应数据}\label{HT-03-04-02-10}
\centering
\begin{tabular}{|c|c|c|c|}
相机类型数量 & 具体内容       & 地址                  & 数据  \\
飞行相机   & CCD面阵-6个     & HT-03-04-02-10-01    & 01-06   \\
上视相机   & CCD面阵-2个     & HT-03-04-02-10-02    & 01-02   \\
下视相机   & CCD面阵-1个     & HT-03-04-02-10-03    & 01-01
\end{tabular}
\end{table}

\newpage%不加这个目录会错
\bibliographystyle{IEEEtran}
\bibliography{reference}
\end{CJK}
\end{document}

%This file should be compiled with latex!!!
